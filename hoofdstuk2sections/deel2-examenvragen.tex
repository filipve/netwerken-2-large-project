\subsection{Examenvragen hoofdstuk 2}
\begin{enumerate}
\item Waarom dat P2P beter is in het distruburen van files dan client-server

\item Prefetchen uitleggen

\item Alle voordelen en nadelen van UDP en TCP uitleggen, en 2 voorbeelden geven waarom/wanneer je gebruik zou maken van UDP of TCP.

\item Waarom maakt een website gebruikt van cookies. Leg uit met behulp van een schema.

\item Je verstuurt een mail via de server mx debbie, zeg hier bij wat er gebeurt en maak gebruik van de volgende termen: smtp, DNS protocol, routering, ARP protocol, client-server model.

\item Zet wat er gebeurt als je naar www.khleuven.be surft met je browser en gebruik hierbij de volgende termen: HTTP protocol, DNS protocol, routering, ARP protocol.

\item Een ISP heeft er belang bij om een peering overeenkomst te hebben met elke andere ISP?

\item waar of niet waar + leg uit (hierop krijg je punten)

    \begin{enumerate}
        \item In de HTTP header field byterange krijgt de gebruiker een indicatie van de grootte van het object.

        \item Voorbeelden van applicatie gateway zijn web cache en een server

        \item Als je zo snel mogelijk een bestand wil versturen naar een verre locatie (+15 hops) dan gebruik je TCP

        \item Een isp moet betalen voor zijn internetverbinding

        \item Er zijn minder regels bij een packetfilter om een bepaalde TCP communicatie toe te staan tussen client-server dan bij stateful inspection firewall

\end{enumerate}

\item Termen uitleggen:

    \begin{enumerate}
         \item Peering

        \item Persistent http

        \item BGP (Border Gateway Protocol)

        \item RTP (Realtime Transport Protocol)

        \end{enumerate}
\end{enumerate}


%\enumeratext{} zorgt ervoor dat je tekst kan typen tussen lijstitems in
%\enumeratext{}

\subsection{vragen uit boek hoofdstuk 2}

SECTION 2.1

\begin{enumerate}

\item 

\enumeratext{

}


\item 

\enumeratext{

}


\item 

\enumeratext{

}


\item 

\enumeratext{

}


\item 

\enumeratext{

}


\item 

\enumeratext{

}


\item 

\enumeratext{

}


\item 

\enumeratext{

}


\item 

\enumeratext{

}


\item 

\enumeratext{

}


\item 

\enumeratext{

}


\item 

\enumeratext{

}



\end{enumerate}

SECTION 2.1

\begin{enumerate}

\item List five nonproprietary Internet applications and the application-layer protocols that they use.

\enumeratext{
The Web: HTTP; 
file transfer: FTP; 
remote login: Telnet; 
Network News: NNTP;
e-mail: SMTP. 
}


\item What is the difference between network architecture and application architecture?

\enumeratext{Network architecture refers to the organization of the communication process into layers (e.g., the five-layer Internet architecture). Application architecture, on the
other hand, is designed by an application developer and dictates the broad structure of the application (e.g., client-server or P2P)}


\item For a communication session between a pair of processes, which process is the client and which is the server?

\enumeratext{The process which initiates the communication is the client; 
the process that waits to be contacted is the server. 

}


\item For a P2P file-sharing application, do you agree with the statement, “There isno notion of client and server sides of a communication session”? Why or why not?

\enumeratext{ No. As stated in the text, all communication sessions have a client side and a server side. In a P2P file-sharing application, the peer that is receiving a file is
typically the client and the peer that is sending the file is typically the server.}


\item What information is used by a process running on one host to identify a process running on another host?

\enumeratext{The IP address of the destination host and the port number of the destination socket. 

}


\item Suppose you wanted to do a transaction from a remote client to a server as fast as possible. Would you use UDP or TCP? Why?

\enumeratext{ You would use UDP. With UDP, the transaction can be completed in one roundtrip time (RTT) - the client sends the transaction request into a UDP socket, and the server sends the reply back to the client's UDP socket. With TCP, a minimum of two RTTs are needed - one to set-up the TCP connection, and another for the client to send the request, and for the server to send back the reply

}


\item Referring to Figure 2.4, we see that none of the applications listed in Figure 2.4 requires both no data loss and timing. Can you conceive of an application that requires no data loss and that is also highly time-sensitive?

\enumeratext{There are no good examples of an application that requires no data loss and timing. If you know of one, send an e-mail to the authors.}


\item List the four broad classes of services that a transport protocol can provide.For each of the service classes, indicate if either UDP or TCP (or both) provides such a service

\enumeratext{

a) Reliable data transfer:
   TCP provides a reliable byte-stream between client and server but UDP does not.

b) A guarantee that a certain value for throughput will be maintained

Neither

c) A guarantee that data will be delivered within a specified amount of time 

Neither

d) Security
   Neither 

}

\item Recall that TCP can be enhanced with SSL to provide process-to-process security services, including encryption. Does SSL operate at the transport layer or the application layer? If the application developer wants TCP to be enhanced with SSL, what does the developer have to do?

\enumeratext{SSL operates at the application layer. The SSL socket takes unencrypted data from the application layer, encrypts it and then passes it to the TCP socket. If the application developer wants TCP to be enhanced with SSL, she has to include the SSL code in the application. }


\end{enumerate}

SECTIONS 2.2–2.5

\begin{enumerate}

\item What is meant by a handshaking protocol?

\enumeratext{A protocol uses handshaking if the two communicating entities first exchange control packets before sending data to each other. SMTP uses handshaking at the
application layer whereas HTTP does not.}


\item Why do HTTP, FTP, SMTP, and POP3 run on top of TCP rather than on UDP?

\enumeratext{The applications associated with those protocols require that all application data be received in the correct order and without gaps. TCP provides this service whereas UDP does not.}


\item Consider an e-commerce site that wants to keep a purchase record for each of its customers. Describe how this can be done with cookies

\enumeratext{
When the user first visits the site, the site returns a cookie number. This cookie number is stored on the user’s host and is managed by the browser. During each subsequent visit (and purchase), the browser sends the cookie number back to the site. Thus the site knows when this user (more precisely, this browser) is visiting the site. 

}


\item Describe how Web caching can reduce the delay in receiving a requested object. Will Web caching reduce the delay for all objects requested by a user or for only some of the objects? Why?

\enumeratext{Web caching can bring the desired content “closer” to the user, perhaps to the same LAN to which the user’s host is connected. Web caching can reduce the delay for all objects, even objects that are not cached, since caching reduces the
traffic on links.}


\item Telnet into a Web server and send a multiline request message. Include in the request message the If-modified-since: header line to force a response message with the 304 Not Modified status code.

\enumeratext{

Issued the following command (in Windows command prompt) followed by the HTTP GET message to the “utopia.poly.edu” web server:


$>$ telnet utopia.poly.edu 80

Since the index.html page in this web server was not modified since Fri, 18 May 2007 09:23:34 GMT, the following output was displayed when the above commands were issued on Sat, 19 May 2007. Note that the first 4 lines are the GET message and header lines input by the user and the next 4 lines (starting from HTTP/1.1 304 Not Modified) is the response from the web server.}


\item  Why is it said that FTP sends control information “out-of-band”?

\enumeratext{FTP uses two parallel TCP connections, one connection for sending control information (such as a request to transfer a file) and another connection for actually transferring the file. Because the control information is not sent over the same connection that the file is sent over, FTP sends control information out of band.}

\item  Suppose Alice, with a Web-based e-mail account (such as Hotmail or gmail), sends a message to Bob, who accesses his mail from his mail server using POP3. Discuss how the message gets from Alice’s host to Bob’s host. Be sure to list the series of application-layer protocols that are used to move the message
between the two hosts

\enumeratext{Message is sent from Alice’s host to her mail server over HTTP. Alice’s mail server then sends the message to Bob’s mail server over SMTP. Bob then transfers the message from his mail server to his host over POP3.}


\item  Print out the header of an e-mail message you have recently received. How many Received: header lines are there? Analyze each of the header lines in the message.

\enumeratext{

}


\item From a user’s perspective, what is the difference between the download-and delete mode and the download-and-keep mode in POP3?

\enumeratext{

With download and delete, after a user retrieves its messages from a POP server,the messages are deleted. This poses a problem for the nomadic user, who may want to access the messages from many different machines (office PC, home PC, etc.). In the download and keep configuration, messages are not deleted after the user retrieves the messages. This can also be inconvenient, as each time the user retrieves the stored messages from a new machine, all of non-deleted messages will be transferred to the new machine (including very old messages).}


\item Is it possible for an organization’s Web server and mail server to have exactly the same alias for a hostname (for example, foo.com)? What would be the type for the RR that contains the hostname of the mail server?

\enumeratext{Yes an organization’s mail server and Web server can have the same alias for ahost name. The MX record is used to map the mail server’s host name to its IP address. }


\item Look over your received emails, and examine the header of a message sent from a user with an .edu email address. Is it possible to determine from the header the IP address of the host from which the message was sent? Do the same for a message sent from a gmail account. 

\enumeratext{

}



\end{enumerate}

SECTION 2.6

\begin{enumerate}

\item In BitTorrent, suppose Alice provides chunks to Bob throughout a 30-second interval. Will Bob necessarily return the favor and provide chunks to Alice in this same interval? Why or why not?

\enumeratext{It is not necessary that Bob will also provide chunks to Alice. Alice has to be in the top 4 neighbors of Bob for Bob to send out chunks to her; this might not occur even if Alice is provides chunks to Bob throughout a 30-second interval.}


\item Consider a new peer Alice that joins BitTorrent without possessing any chunks. Without any chunks, she cannot become a top-four uploader for any of the other peers, since she has nothing to upload. How then will Alice get her first chunk?

\enumeratext{ Alice will get her first chunk as a result of she being selected by one of her neighbors as a result of an “optimistic unchoke,” for sending out chunks to her.}

\item What is an overlay network? Does it include routers? What are the edges in the overlay network?

\enumeratext{The overlay network in a P2P file sharing system consists of the nodes participating in the file sharing system and the logical links between the nodes. There is a logical link (an “edge” in graph theory terms) from node A to node B if there is a semi-permanent TCP connection between A and B. An overlay network does not include routers. With Gnutella, when a node wants to join the Gnutella network, it first discovers (“out of band”) the IP address of one or more nodes already in the network. It then sends join messages to these nodes. When the node receives confirmations, it becomes a member of the of Gnutella network. Nodes maintain their logical links with periodic refresh messages.}

\item Consider a DHT with a mesh overlay topology (that is, every peer tracks all peers in the system). What are the advantages and disadvantages of such a design? What are the advantages and disadvantages of a circular DHT (with no shortcuts)?

\enumeratext{
Mesh DHT: The advantage is to a route a message to the peer closest to the key,
only one hop is required; the disadvantage is that each peer must track all other
peers in the in the DHT. Circular DHT: the advantage is that each peer needs to
track only a few other peers; the disadvantage is that O(N) hops are needed to
route a message to a peer responsible for the key. 

}


\item List at least four different applications that are naturally suitable for P2P architectures. (Hint: File distribution and instant messaging are two.)

\enumeratext{
 a) File Distribution
 b) Instant Messaging
 c) Video Streaming
 d) Distributed Computing 

}




\end{enumerate}

SECTION 2.7

\begin{enumerate}

\item In Section 2.7, the UDP server described needed only one socket, whereas the TCP server needed two sockets. Why? If the TCP server were to support n simultaneous connections, each from a different client host, how many sockets would the TCP server need?

\enumeratext{ With the UDP server, there is no welcoming socket, and all data from different clients enters the server through this one socket. With the TCP server, there is a welcoming socket, and each time a client initiates a connection to the server, a new socket is created. Thus, to support n simultaneous connections, the server would need n+1 sockets. }

\item For the client-server application over TCP described in Section 2.7, why must the server program be executed before the client program? For the client-server application over UDP, why may the client program be executed before the server program?

\enumeratext{For the TCP application, as soon as the client is executed, it attempts to initiate a TCP connection with the server. If the TCP server is not running, then the client
will fail to make a connection. For the UDP application, the client does not initiate connections (or attempt to communicate with the UDP server) immediately upon execution}



\end{enumerate}

Problems : True or false?

\begin{enumerate}

\item A user requests a Web page that consists of some text and three images. For this page, the client will send one request message and receive four response messages.

\enumeratext{False}


\item Two distinct Web pages (for example, www.mit.edu/research.html and www.mit.edu/students.html) can be sent over the same persistent connection.

\enumeratext{True}


\item With nonpersistent connections between browser and origin server, it is possible for a single TCP segment to carry two distinct HTTP request messages.

\enumeratext{False}


\item The Date: header in the HTTP response message indicates when the object in the response was last modified

\enumeratext{False}

\item  HTTP response messages never have an empty message body.

\enumeratext{True}




\end{enumerate}
