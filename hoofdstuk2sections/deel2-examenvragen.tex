\subsection{Examenvragen hoofdstuk 2}
\begin{enumerate}
\item Waarom dat P2P beter is in het distruburen van files dan client-server

\item Prefetchen uitleggen

\item Alle voordelen en nadelen van UDP en TCP uitleggen, en 2 voorbeelden geven waarom/wanneer je gebruik zou maken van UDP of TCP.

\item Waarom maakt een website gebruikt van cookies. Leg uit met behulp van een schema.

\item Je verstuurt een mail via de server mx debbie, zeg hier bij wat er gebeurt en maak gebruik van de volgende termen: smtp, DNS protocol, routering, ARP protocol, client-server model.

\item Zet wat er gebeurt als je naar www.khleuven.be surft met je browser en gebruik hierbij de volgende termen: HTTP protocol, DNS protocol, routering, ARP protocol.

\item Een ISP heeft er belang bij om een peering overeenkomst te hebben met elke andere ISP?

\clearpage

\item waar of niet waar + leg uit (hierop krijg je punten)

    \begin{enumerate}
        \item 2.7.8.1	In de HTTP header field byterange krijgt de gebruiker een indicatie van de grootte van het object.

        \item 2.7.8.2	Voorbeelden van applicatie gateway zijn web cache en een server

        \item 2.7.8.3	Als je zo snel mogelijk een bestand wil versturen naar een verre locatie (+15 hops) dan gebruik je TCP

        \item 2.7.8.4	Een isp moet betalen voor zijn internetverbinding

        \item 2.7.8.5	Er zijn minder regels bij een packetfilter om een bepaalde TCP communicatie toe te staan tussen client-server dan bij stateful inspection firewall

\end{enumerate}

\clearpage

\item Termen uitleggen:

    \begin{enumerate}
         \item Peering

        \item Persistent http

        \item BGP (Border Gateway Protocol)

        \item TP (Realtime Transport Protocol)

        \end{enumerate}
\end{enumerate}


%\enumeratext{} zorgt ervoor dat je tekst kan typen tussen lijstitems in
%\enumeratext{}
