\subsection{Principes van netwerk applicaties}

\subsubsection{Structuren van netwerkapplicaties}

\begin{multicols}{2}
  \begin{itemize}
      \item e-mail
 \item web
 \item text messaging
 \item remote login
 \item P2P file sharing
 \item multi-user network games
 \item streaming stored video (YouTube, Hulu, Netflix)
 \item voice over IP (e.g., Skype)
 \item real-time video conferencing
 \item social networking
 \item search
 \item ...
  \end{itemize}
\end{multicols}

\noindent \textbf{Cliënt-serverstructuur:} In deze structuur is er een host die altijd aan is (de server). Deze host behandelt de verzoeken van zijn clients. De server heeft een vast IP-adres en is dus steeds bereikbaar. Probleem? Als er slechts 1 server is, kan deze overbelast geraken. Oplossing? Het gebruik van datacenters (verschillende servers die samen 1 server vormen).\\\\
\textbf{P2P-structuur (peer-to-peer)}: Hosts maken gebruik van rechtstreekse communicatie met elkaar. Pluspunt: \textbf{zelfschaalbaarheid}, elke peer die wordt toegevoegd zorgt voor meer capaciteit. P2P is kosten efficiënt, omdat ze geen significante server infrastructuur en bandbreedte nodig hebben.\\\\
$\Rightarrow$ Dus geen server die altijd aanstaat.\\
$\Rightarrow$ Peers zijn met wederkerend optreden geconnecteerd en veranderen van IP-adres.\\\\
\textbf{Peers} = periodiek met elkaar verbonden hosts waarbij rechtstreekse communicatie toegepast wordt.\\\\
- Dit zijn gewone gebruikers van desktops en laptops.\\\\
Toekomstige P2P-applicaties moeten rekening houden met 3 punten:
\begin{enumerate}
\item \textbf{ISP-vriendelijkheid}: De meeste ISP’s die thuistoegang tot het internet verlenen gaan ervan uit dat het internet asymmetrisch gebruikt zal worden (meer downloaden dan uploaden). Aangezien er bij P2P veel moet geüpload worden, moet de applicatie hier rekening bij houden.
\item \textbf{Beveiliging}: Aangezien de P2P-applicaties zo toegankelijk zijn, moeten ze goed beveiligd zijn.
\item \textbf{Prikkels}: Er moeten mensen zijn die hun upload-capaciteit willen opgeven voor de applicatie (seeden).
\end{enumerate}

\subsubsection{Netwerk applicatie architectuur}

De applicatie architectuur, is ontworpen door een applicatie ontwikkelaar en bepaalt hoe de applicatie gestructureerd is over de verschillende systemen.

\newpage \noindent Hoewel, P2P applicaties staan voor drie grote uitdagingen:

\subsubsection{Processen communiceren}

\textbf{Proces}: Een applicatie die draait op een host. (Communicatie tussen processen op 1 host = \textbf{interprocescommunicatie})
\\\\
Processen in 2 verschillende hosts communiceren met elkaar door \textbf{berichten} uit te wisselen over een computernetwerk.

\subsubsubsection{Client en server processen}

\textbf{client proces}: het proces dat de communicatie start\\\\
\textbf{server proces}: proces dat wacht om gecontacteerd te worden.

\subsubsubsection{De interface tussen het proces en het computer netwerk}

\textbf{Socket}: een proces verzendt berichten naar en ontvangt berichten van het netwerk via een netwerkinterface.\\\\
Applicaties communiceren over een netwerk via sockets (= \textbf{Application Programming Interface = API)}\\

\noindent Een socket is de interface tussen de applicatie laag, en de transport laag binnen een host.

\subsubsection{Transport services beschikbaar voor applicaties}

Wat zijn de diensten dat een transport laag protocol kan aanbieden aan applicaties? We kunnen dit breed classificeren in vier dimensies.

\subsubsubsection{Reliable Data Transfer = Betrouwbare gegevensoverdracht}

Voorziet of een dienst volledige zekerheid kan bieden of pakketten zullen aankomen of niet. Als men dit aanbiedt voldoet men aan reliable data transfer.(het is mogelijk dat dit niet zo belangrijk is bv. Skype waar er af en toe een pakket op de verbinding mag wegvallen, de gebruiker blijft verstaanbaar)

\subsubsubsection{Throughput = doorvoercapaciteit (ongeveer hetzelfde als bandbreedte)}
= de snelheid waarin het zendende proces de bits kan leveren aan het ontvangende proces. Omdat andere sessies de bandbreedte delen, zal de beschikbare throughput over tijd schommelen. (r bits/sec)\\\\
Applicaties die througput vereisten hebben zijn \textbf{bandwidth-sensitive applicaties}. \\\\
\textbf{Elastic applicaties} kunnen gebruik maken als veel of weinig throughput er op dat moment beschikbaar is. Natuurlijk hoe meer througput, hoe beter.

\subsubsubsection{Timing}

Een transport protocol kan ook tijd garanties voorzien. Dit kan ook in vele vormen voorkomen. Een garantie kan zijn dat elke bit die de zender stuurt niet later dan 100 msec mag aankomen. Zo’n service is aantrekkelijk voor interactieve real-time applicaties. Voor non-real-time applicaties is lager vertraging altijd de voorkeur op hogere vertraging, maar er zijn geen vaste beperkingen geplaatst op de vertragingen.

\subsubsubsection{Security}
$\Rightarrow$ encryptie ja/nee?\\
$\Rightarrow$ data integriteit\\
$\Rightarrow$ authenticatie.

\subsubsection{Transport diensten die door het internet voorzien worden}

Het internet maakt twee transport protocollen beschikbaar voor applicaties, namelijk UDP en TCP.

\subsubsubsection{TCP diensten}

Het TCP service model bevat een connectie georiënteerde dienst en een betrouwbare data transfer dienst. Wanneer een applicatie TCP oproept als zijn transport protocol, krijgt de applicatie volgende diensten van TCP.
\begin{itemize}
    \item \textbf{Connection-oriented service}. 
    Er gebeurt eerst een handshake en er wordt nadien een full-duplex verbinding gestart tussen 2 hosts. Wanneer de applicatie eindigt met het zenden van berichten, moet het de connectie afbreken.
    \item Reliable data transfer service. De communicerende processen kunnen op TCP vertrouwen om alle data die gezonden is zonder een fout en in de juiste volgorde te leveren.
    \item Congestie-controlemechanisme: verwittig de zender wanneer het netwerk overbelast is.
\end{itemize}

\noindent TCP heeft niet:

\bi
\itf Timing
\itf minimum throughput guarantee
\itf security
\ei

\subsubsubsection{UDP diensten}

De reden dat er soms voor UDP wordt gekozen is omdat er door het gebrek van veiligheid met een veel hogere snelheid kan gewerkt worden. (Minimale snelheid is nodig voor bv. Skype) Veel firewalls blokkeren echter UDP, dus wordt er ook een alternatief verzien om met TCP te kunnen werken.

UDP IS:

\bi
\itf lichtgewicht transport protocol
\itf connectie-loos $\Rightarrow$ geen handshake die voorafgaat 
\ei

\noindent UDP $\Rightarrow$ voorziet een onbetrouwbare data transfer service.\\
$\Rightarrow$ Het garandeert niet dat het bericht op de juiste manier zal aankomen.\\

\subsubsection{Applicatie laag protocollen}

Een applicatie laag protocol definieert hoe een applicatie zijn processen berichten doorgeven aan elkaar. In het bijzonder, een applicatie laag protocol definieert:
\begin{itemize}
    \item De types van berichten die uitgewisseld worden (request en response berichten)
    \item De syntax van de verschillende bericht types, zoals de velden in het bericht en hoe de velden afgebakend zijn.
    \item De betekenis van de velden, de betekenis van de informatie in de velden.
    \item Regels voor het bepalen wanneer en hoe een proces berichten stuurt en antwoord op berichten.
\end{itemize}
Sommige applicatie laag protocollen zijn gespecificeerd in RFCs (Requests For Comments) en zijn hierdoor in het publieke domein. Het is belangrijk om een onderscheid te maken tussen netwerk applicaties en applicatie laag protocollen. Een applicatie laag protocol is enkel een stukje van een netwerk applicatie.
