\subsection{Principes van netwerk applicaties}

De kern van netwerk applicatie ontwikkeling is het schrijven van programma’s die uitgevoerd kunnen worden op verschillende systemen en met elkaar kunnen communiceren over het netwerk.
Er moet geen software geschreven worden die wordt uitgevoerd op netwerkkern apparaten (routers, switches). Beperken van applicatiesoftware op het eindsysteem heeft het vergemakkelijkt voor een snelle ontwikkeling en inzetting van netwerk applicaties.

\subsubsection{Netwerk applicatie architectuur}

Vanuit het perspectief van de applicatie ontwikkelaar, staat de network architectuur vast en voorziet een specifieke verzameling van diensten aan applicaties. De applicatie architectuur, is ontworpen door een applicatie ontwikkelaar en bepaald hoe de applicatie gestructureerd is over de verschillende systemen. In het kiezen voor de architectuur, zal de ontwikkelaar waarschijnlijk één van de twee overheersende architecturale paradigma teken die gebruikt worden in huidige netwerken. De client-server of P2P architectuur.
In een client-server architectuur, is er altijd aanstaande host, die de server genoemd wordt, die request behandeld van vele andere hosts, clients genaamd. Merk op dat met deze architectuur, de clienten niet direct communiceren met elkaar. Een ander kenmerk is dat de server een vast gekende adres heeft. Vaak in een client-server applicatie, is een enkele server host niet in staat om alle request van de clienten bij te houden. Daarom wordt er een data center, die groot aantal hosts bevat, gebruikt om zo een krachtige virtuele server te maken.
In een P2P architectuur, is er minimale of geen vertrouwen op gededicteerde servers in de data centers. In plaats hiervan buit de applicatie directe communicatie tussen paren van even connecterende hosts, peers genaamd, uit. De peers worden niet beheerd voor een service provider, maar in plaats daarvan desktops en laptops die gecontroleerd zijn door gebruikers.
Een van de meest dwingende kenmerken van P2P is hun self scalability. Elke peer voegt service capaciteit toe aan het systeem door bestanden te verdelen naar andere peers. P2P is kosten efficiënt, omdat ze geen significante server infrastructuur en bandbreedte nodig hebben.
\newpage \noindent Hoewel, P2P applicaties staan voor drie grote uitdagingen:

\begin{enumerate}
    \item ISP friendly. Meeste residentiële ISPs, zijn gedimensioneerd voor asymmetrische bandbreedte gebruik, dat is, voor veel meer downstream dan upstream verkeer.
    \item Security. Door hun veel gedeelde en open karakter, kunnen P2P applicaties een uitdaging zijn om te beveiligen.
    \item Incentives (Prikkels). Het succes van P2P applicaties hangt ook er van af op gebruikers te overtuigen om bandbreedte, opslagruimte en resources aan de applicatie te geven.
\end{enumerate}


\subsubsection{Processen communiceren}

Processen op twee verschillende systemen communiceren met elkaar door het uitwisselen van berichten over het netwerk. Een zendend proces maakt en zend berichten op het netwerk. Een ontvangend proces ontvangt deze berichten en kan antwoorden door berichten terug te sturen.

\subsubsubsection{Client en server processen}

Een netwerk applicatie bestaat uit een paar van processen die berichten naar elkaar sturen over een netwerk. We kunnen de client en server processen als volgt definiëren.
In de context van een communicatie sessie tussen een paar van processen, het proces die de communicatie begint, wordt bestempeld als de client. Het proces die wacht om gecontacteerd te worden om een sessie te beginnen is de server.

\subsubsubsection{De interface tussen het proces en het computer netwerk}

Elk bericht de verzonden wordt moet door een onderliggend netwerk gaan. Een proces zend berichten in, en krijgt berichten uit, het netwerk door een software interface die socket genoemd wordt. Een socket is de interface tussen de applicatie laag, en de transport laag binnen een host. Dit wordt ook wel de Applicatie Programming Interface (API) genoemd tussen de applicatie en het netwerk, sind de socket een programming interface is waarmee netwerk applicaties gebouwd zijn. De applicatie ontwikkelaar heeft de volledige controle over de applicatie laag kant van de socket, maar heeft heel weinig controle over de transport laag kant. De enige controle dat hem heeft is de keuze van het transport protocol en misschein de mogelijkheid om een paar parameters vast te zetten zoals maximum buffer, segment grootte. 

\clearpage

\subsubsubsection{Addressing processes}

Om pakketten te kunnen sturen van een lopend proces op één host naar een lopend proces op een andere host, moet het ontvangende proces een adres hebben. Om het ontvangende proces te identificeren, moeten er twee delen van informatie gespecificeerd zijn. Het adres van de host en een identificator die het ontvangende proces specificeert op de aankomst host.
Op het internet, is de host geïdentificeerd door zijn IP adres. Naast het weten van het adres van de host, moet het zendende proces ook het ontvangende proces identificeren. Hiervoor dient het poort nummer. Populaire applicaties zijn specifieke poort nummers toegewezen.

\subsubsection{Transport services beschikbaar voor applicaties}

Wat zijn de diensten dat een transport laag protocol kan aanbieden aan applicaties? We kunnen dit breed classificeren in vier dimensies.

\subsubsubsection{Reliable Data Transfer}

Pakketten kunnen verloren gaan binnen een computer netwerk. Dus om applicaties te ondersteunen waar er garantie moet zijn dat de data die gezonden wordt, volledig en correct geleverd wordt moet er iets gedaan worden. als een protocol zo’n gegarandeerde data leverings dienst voorziet, wordt het gezegd dat het reliable data transfer voorziet. Een belangrijke dienst dat een transport laag protocol potentieel kan voorzien aan een applicatie is van proces naar proces betrouwbare data vervoer. Wanneer dit voorzien is, kan het zendende proces gewoon zijn data in de socket plaatsen en met volledige zekerheid weten dat de data zonder fouten aankomt.
Wanneer een transport laag protocol dit niet voorziet, kunnen sommige van de data die gezonden zijn nooit aankomen. Dit kan acceptabel zijn voor loss-tolerant applicaties, zoals multimedia applicaties. Data die verloren gaan, resulteren in een kleine storing in de audio/video.

\clearpage

\subsubsubsection{Throughput}

In het concept van een communicatie sessie tussen twee processen in een netwerk pad, is througput de snelheid waarin het zendende proces de bits kan leveren aan het ontvangende proces. Omdat andere sessies de bandbreedte delen, zal de beschikbare throughput over tijd schommelen. Dit leid naar een andere dienst die een transport laag protocol kan voorzien, namelijk het voorzien van een gegarandeerde beschikbare throughput op een bepaalde snelheid. Met zo’n service kan de applicatie een gegarandeerde throughput vragen van r bits/sec, en het transport protocol gaat er dan voor zorgen dat de beschikbare throughput minstens r bits/sec is. Applicaties die througput vereisten hebben zijn bandwidth-sensitive applicaties. Veel huidige multimedia applicaties zijn bandbreedte gevoelig, hoewel kunnen sommige applicaties zich aanpassen aan de huidige beschikbare throughput. Elastic applicaties kunnen gebruik maken als veel of weinig throughput er op dat moment beschikbaar is. Natuurlijk hoe meer througput, hoe beter.

\subsubsubsection{Timing}

Een transport protocol kan ook tijd garanties voorzien. Dit kan ook in vele vormen voorkomen. Een garantie kan zijn dat elke bit die de zender stuurt niet later dan 100 msec mag aankomen. Zo’n service is aantrekkelijk voor interactieve real-time applicaties. Voor non-real-time applicaties is lager vertraging altijd de voorkeur op hogere vertraging, maar er zijn geen vaste beperkingen geplaatst op de vertragingen.

\subsubsubsection{Security}

Een transport protocol kan een applicatie voorzien van één of meer beveiligings diensten. Bijvoorbeeld om alle data die verstuurd wordt te encrypteren. Zo’n dienst zou confidentialiteit voorzien tussen twee processen. Het kan ook andere beveiligings diensten voorzien zoals data integriteit en authenticatie.

\subsubsection{Transport diensten die door het internet voorzien worden}

Het internet maakt twee transport protocollen beschikbaar voor applicaties, namelijk UDP en TCP.

\clearpage

\subsubsubsection{TCP diensten}

Het TCP service model bevat een connectie georiënteerde dienst en een betrouwbare data transfer dienst. Wanneer een applicatie TCP oproept als zijn transport protocol, krijgt de applicatie volgende diensten van TCP.
\begin{itemize}
    \item Connection-oriented service. TCP laat de client en server transport laag controle informatie uitwisselen met elkaar voordat de applicatie niveau berichten gaan doorstromen. Deze handshake procedure waarschuwt de client en server, wat hen toestaat om zich voor te bereiden ongekend aantal pakketten. Na de handshake fase, bestaat er een TCP connectie tussen de sockets van de twee processen. De connectie is een full-duplex connectie. Wanneer de applicatie eindigt met het zenden van berichten, moet het de connectie afbreken.
    \item Reliable data transfer service. De communicerende processen kunnen op TCP vertrouwen om alle data die gezonden is zonder een fout en in de juiste volgorde te leveren.
\end{itemize}
TCP bevat ook een congestion-control mechanisme, een dienst voor het algemeen welzijn van het internet. De mechanisme smoort een zendende proces wanneer het netwerk opgehoopt is tussen de zender en ontvanger. Dit mechanisme probeert ook elke TCP connectie te limiteren voor een eerlijk deel van het netwerk zijn bandbreedte.

\subsubsubsection{UDP diensten}

UDP is een lichtgewicht transport protocol, die minimale diensten voorziet. UDP is connectie-loos, dus er is geen handshake die voorafgaat. UDP voorziet een onbetrouwbare data transfer service. Het garandeert niet dat het bericht op de juiste manier zal aankomen. UDP kan data in het netwerk pompen aan de zelf gewenste snelheid.

\subsubsection{Applicatie laag protocollen}

Een applicatie laag protocol definieert hoe een applicatie zijn processen berichten doorgeven aan elkaar. In het bijzonder, een applicatie laag protocol definieert:
\begin{itemize}
    \item De types van berichten die uitgewisseld worden (request en response berichten)
    \item De syntax van de verschillende bericht types, zoals de velden in het bericht en hoe de velden afgebakend zijn.
    \item De betekenis van de velden, de betekenis van de informatie in de velden.
    \item Regels voor het bepalen wanneer en hoe een proces berichten stuurt en antwoord op berichten.
\end{itemize}
Sommige applicatie laag protocollen zijn gespecificeerd in RFCs (Requests For Comments) en zijn hierdoor in het publieke domein. Het is belangrijk om een onderscheid te maken tussen netwerk applicaties en applicatie laag protocollen. Een applicatie laag protocol is enkel een stukje van een netwerk applicatie.
