\documentclass[a4paper,11pt]{article}
\usepackage{import}
\usepackage{example}

\usepackage{makeidx}

\usepackage{showidx} %It prints out all index entries in the left margin of the text.
\usepackage{palatino} 
 %\usepackage{fontspec}
 %\setmainfont{palatino}
 
\makeindex

\begin{document}
\glsaddall
\import{./}{title.tex}

\clearpage
\thispagestyle{empty}

\tableofcontents

\clearpage



%hoofdstuk 2
\setcounter{section}{1}
\section{Applicatie laag}
\import{hoofdstuk2sections/}{deel2-1.tex}
\import{hoofdstuk2sections/}{deel2-2.tex}
\import{hoofdstuk2sections/}{deel2-3.tex}
\import{hoofdstuk2sections/}{deel2-4.tex}
\import{hoofdstuk2sections/}{deel2-5.tex}
\import{hoofdstuk2sections/}{deel2-examenvragen.tex}

%hoofdstuk 7
\setcounter{section}{6}
\section{Multimedia netwerken}
\import{hoofdstuk7sections/}{deel7-1.tex}
\import{hoofdstuk7sections/}{deel7-2.tex}
\import{hoofdstuk7sections/}{deel7-3.tex}
\import{hoofdstuk7sections/}{deel7-4.tex}
\import{hoofdstuk7sections/}{deel7-5.tex}
\import{hoofdstuk7sections/}{deel7-examenvragen.tex}

%hoofdstuk 8
\setcounter{section}{7}
\section{Beveiliging in computernetwerken}
\import{hoofdstuk8sections/}{deel8-1.tex}
\import{hoofdstuk8sections/}{deel8-2.tex}
\import{hoofdstuk8sections/}{deel8-3.tex}
\import{hoofdstuk8sections/}{deel8-4.tex}
\import{hoofdstuk8sections/}{deel8-5.tex}
\import{hoofdstuk8sections/}{deel8-6.tex}
\import{hoofdstuk8sections/}{deel8-7.tex}
\import{hoofdstuk8sections/}{deel8-8.tex}
\import{hoofdstuk8sections/}{deel8-examenvragen.tex}

\clearpage

\begin{comment}
%The \texttt{glossaries} package automatically generates a list of glossary entries. It's great for keeping track of your \gls{domain-knowledge} and \glspl{tla}. In this example we've put the glossary definitions in a separate \texttt{glossary.tex} file, which you can edit via the project menu.

%The \Gls{latex} typesetting markup language is specially suitable 
for documents that include \gls{maths}. \Glspl{formula} are 
rendered properly an easily once one gets used to the commands.
 
%Given a set of numbers, there are elementary methods to compute 
its \acrlong{gcd}, which is abbreviated \acrshort{gcd}. This 
process is similar to that used for the \acrfull{lcm}.

\end{comment}

\printglossaries

\listoffigures


\clearpage
%\import{./}{bibliography.tex}

\end{document}
