
%glossary

%A

\newglossaryentry{ACL (Access Control List)}{
name={Access Control List },
description={ }
}

\newglossaryentry{AES: Advanced Encryption Standard }{
name={AES: Advanced Encryption Standard },
description={ }
}

\newglossaryentry{Authenticatieprotocol}{
name={Authenticatieprotocol},
description={Een protocol dat wordt uitgevoerd voor de gegevensoverdracht om te zien met wie we aan het communiceren zijn}
}

%B

\newglossaryentry{bandbreedtegevoelige applicaties}{
name={bandbreedtegevoelige applicaties },
description={Applicaties die veel gebruik maken van doorvoercapaciteit }
}

\newglossaryentry{berichtintegriteit}{
name={berichtintegriteit},
description={Wanneer een bericht “gehasht” is, mag het niet mogelijk zijn om het door een ander gehasht bericht te vervangen }
}

\newglossaryentry{Best-effort}{
name={Best-effort},
description={IP doet zijn uiterste best om elk datagram zo snel mogelijk van de bron naar de bestemming te verplaatsen, maar geen rekening houdt met vertragingen of pakketten die verloren gaan.  }
}

\newglossaryentry{BGP (Border Gateway Protocol)}{
name={BGP (Border Gateway Protocol)},
description={ }
}

\newglossaryentry{Blanket coverage = beveiligingsdeken }{
name={Blanket coverage = beveiligingsdeken },
description={Alle gegevens die van de ene naar de andere entiteit worden verzonden kunnen niet door derden ontsleuteld worden. }
}

\newglossaryentry{Block cipher }{
name={Block cipher},
description={ }
}

\newglossaryentry{Bronrecords}{
name={Bronrecords},
description={records waarin de hostnaam met bijbehorend IP staat opgeslagen (samen met TTL en Type)}
}


%C

\newglossaryentry{CA = certificatie Authoriteit }{
name={CA = certificatie Authoriteit },
description={Deze bindt een publieke sleutel aan een particuliere Entiteit, E. Authenticeert de echtheid van identiteiten en geeft certificaten uit.}
}

\newglossaryentry{cache}{
name={cache },
description={specify date of cached copy in HTTP request }
}

\newglossaryentry{CDN = Content distribution networks }{
name={CDN = Content distribution networks},
description={Een netwerk van servers die zijn verspreid over meerdere locaties waarop kopieën staan van de video’s (bv. YouTube)}
}

\newglossaryentry{chunks  }{
name={chunks},
description={kleine deeltjes van het torrent bestand }
}

\newglossaryentry{Cipher-block chaining }{
name={Cipher-block chaining },
description={Een techniek die gebruikt wordt om encryptie toe te passen bij berichten die verzonden worden.  }
}

\newglossaryentry{ciphertext}{
name={ciphertext},
description={versleutelde tekst }
}

\newglossaryentry{client process}{
name={client process },
description={process that initiates communication }
}

\newglossaryentry{Conditional GET }{
name={Conditional GET },
description={Een GET-bericht dat de proxy naar de server stuurt om te achterhalen of het bestand dat in zijn cache staat al een oud bestand is of niet, dus of het cache een up-to-date versie heeft of niet}
}
\newglossaryentry{Connection closure}{
name={Connection closure},
description={Speciale berichten om de connectie veilig te sluiten}
}

\newglossaryentry{Continue weergave }{
name={Continue weergave },
description={De frames moeten ingeladen zijn vóór dat ze worden weergegeven anders resulteert dit in haperingen. }
}

\newglossaryentry{Cookie}{
name={Cookie},
description={bestand dat wordt bijgehouden door de server waarin bepaalde gegevens staan opgeslagen i.v.m. met de gast op zijn webpagina}
}



%D





\newglossaryentry{denialofservice}{
name={Denial of service},
description={ Service voorkomen zodat het niet door anderen kan gebruikt worden. (vb. resources overbelasten) }
}

\newglossaryentry{Diffie-Hellman key echange algoritm }{
name={Diffie-Hellman },
description={ }
}

\newglossaryentry{Digitale handtekening }{
name={Digitale handtekening },
description={Dit is een cryptografische techniek vergelijkend met een handtekening zetten. }
}

\newglossaryentry{DMZ = demilitarized zone }{
name={DMZ},
description={ }
}

\newglossaryentry{DNS}{
name={DNS },
description={Een gedistribueerde database die is geïmplementeerd in een hiërarchie van DNS-servers}
}

\newglossaryentry{DNS-caching}{
name={DNS-caching},
description={Een DNS-server kan wanneer deze een DNS-antwoord ontvangt, de verwijzing in het lokale cachegeheugen plaatsen. (Wordt vooral gebruikt bij recursieve verzoeken)}
}

\newglossaryentry{Dynamic Adaptive Streaming over HTTP (DASH) }{
name={Dynamic Adaptive Streaming over HTTP (DASH)},
description={De video wordt gecodeerd in verschillende uitvoeringen waarbij elke versie een andere bitsnelheid heeft en dus ook een andere kwaliteit (240p, 360p bij YouTube)}
}







%E





\newglossaryentry{elastische applicaties. }{
name={elastische applicaties. },
description={Applicaties die niet veel gebruik maken van doorvoercapaciteit }
}

\newglossaryentry{enchilada }{
name={enchilada },
description={een versleuteld IPv4 datagram met een ESP-trailerveld en vooraan een ESP-header }
}

\newglossaryentry{End-to-end-delay}{
name={End-to-end-delay},
description={de totale transmissievertraging. Vertraging > 150ms is onhoorbaar door mens. 150 < Vertraging > 400 is acceptabel. Vertraging > 400 kan interactie sterk verhinderen}
}



%F


\newglossaryentry{Forward Error Correction (FEC) }{
name={Forward Error Correction (FEC) },
description={Toevoegen van redundante informatie aan de oorspronkelijke packetstream}
}

\newglossaryentry{FTP Besturingsverbinding }{
name={FTP Besturingsverbinding },
description={Wordt gebruikt om controle-informatie tussen de twee hosts uit te wisselen (gebruikersnaam, wachtwoord, PUT, GET, ...) }
}

\newglossaryentry{FTP Gegevensverbinding }{
name={FTP Gegevensverbinding  },
description={Wordt gebruikt voor het verzenden van het eigenlijke bestand. Het bestand wordt op een andere verbinding verzonden als de controle-informatie en wordt dus out-of-band genoemd.}
}

\newglossaryentry{FTP in passive mode}{
name={FTP passive mode},
description={ }
}


%G




%H


\newglossaryentry{HTTP}{
name={HTTP},
description={ }
}

\newglossaryentry{http hijacking }{
name={HTTP Hijacking},
description={ }
}



%I

\newglossaryentry{IMAP = Internet Mail Access Protocol  }{
name={IMAP = Internet Mail Access Protocol},
description={pullprotocol dat dient om mails op de server te categoriseren en in mappen te steken en houdt sessie-informatie bij  }
}

\newglossaryentry{initialisatievector (IV) }{
name={initialisatievector (IV) },
description={Voordat de verzender het bericht encrypteert, genereert hij een willekeurige waarde van x-aantal bits }
}

\newglossaryentry{Interactiviteit }{
name={Interactiviteit},
description={Dit biedt de mogelijkheid dat de gebruiker de video kan pauzeren, terugspoelen en verder spoelen. }
}

\newglossaryentry{Interleaving}{
name={Interleaving},
description={Het komt er hier op neer dat wanneer men geen redundantie wilt creëeren, men gebruik gaat maken van interleaving. Dit is de originele stream dooreenhalen volgens een vast patroon en deze dan terug in elkaar te boksen na packetverlies. De hierdoor ontstaan er hele kleine onderbrekingen. Nadeel: wachttijd. }
}

\newglossaryentry{Internet exchange }{
name={Internet exchange },
description={ }
}


\newglossaryentry{Internet Key Exchange-protocol (IKE)}{
name={Internet Key Exchange-protocol (IKE)},
description={geautomatiseerd mechanisme voor het maken van beveiligingsassociaties}
}

\newglossaryentry{interprocescommunicatie }{
name={interprocescommunicatie },
description={Communicatie tussen processen op 1 host}
}

\newglossaryentry{IP-anycast }{
name={	IP-anycast},
description={laat internetrouters het beste pad bepalen}
}

\newglossaryentry{IPsec }{
name={IPsec },
description={IP-securityprotocol => zorgt voor beveiliging op de netwerklaag. }
}

%J



\newglossaryentry{Jitter}{
name={Jitter},
description={Het fenomeen dat zich voordoet wanneer vertragingen tussen het verzenden en ontvangen van pakketten varieert.  Oplossing: volgnummers, tijdstempels, weergavevertraging.  }
}




%K

\newglossaryentry{Ticket (Kerberos) }{
name={Kerberos},
description={ }
}

\newglossaryentry{key derivation }{
name={key derivation},
description={verkrijgen van een sleutel}
}



%L


%M



\newglossaryentry{Mailbox }{
name={Mailbox },
description={contains incoming messages for user }
}

\newglossaryentry{manifestbestand}{
name={manifestbestand},
description={bestand dat URL en bijbehorende bitsnelheid levert voor elke versie van de video }
}

\newglossaryentry{Message Authentication Code (MAC)  }{
name={Message Authentication Code (=MAC)  },
description={berichtauthenticatiecode }
}

%N

\newglossaryentry{non persistente verbinding }{
name={non persistente verbinding },
description={Verzoeken en responsen lopen via verschillende TCP verbinding(en)}
}

\newglossaryentry{Nonce }{
name={Nonce },
description={nummer (R) dat maar 1 keer in je leven gebruikt wordt. }
}


%O

\newglossaryentry{Overlaynetwerk}{
name={Overlaynetwerk},
description={virtueel netwerk dat gemaakt wordt bovenop het bestaande netwerk tussen de peers}
}


%P

\newglossaryentry{P2P-structuur (peer-to-peer): }{
name={P2P-structuur (peer-to-peer): },
description={Hosts maken gebruik van rechtstreekse communicatie met elkaar}
}

\newglossaryentry{payload}{
name={payload},
description={Enchilada met een MAC toegevoegd achteraan}
}

\newglossaryentry{Peering}{
name={Peering},
description={ }
}

\newglossaryentry{peers}{
name={peers },
description={periodiek met elkaar verbonden hosts waarbij rechtstreekse communicatie toegepast wordt}
}


\newglossaryentry{Persistent http}{
name={Persistent http},
description={ }
}

\newglossaryentry{persistente verbinding }{
name={persistente verbinding },
description={Wanneer verzoeken en responsen via dezelfde TCP verbinding(en) lopen }
}

\newglossaryentry{PGP (Pretty Good Privacy) }{
name={pgp},
description={Dit is een e-mailversleutelingsmethode dat nu een standaard geworden is. }
}


\newglossaryentry{plain text}{
name={plain text},
description={gewone niet versleutelde tekst }
}

\newglossaryentry{playbackaanval}{
name={playbackaanval},
description={Sleutel kan onderschept worden en zo kan wachtwoord gekraakt worden}
}

\newglossaryentry{POP3 = Post Office Protocol }{
name={POP3 = Post Office Protocol},
description={pullprotocol voor mails}
}


\newglossaryentry{Prefetchen }{
name={Prefetchen},
description={zelfde principe als grijze laadbalk bij YouTube. Je laadt data in op voorhand zodat de video kan afspelen en je niet moet wachten om de video te kijken. sectie: streamen met HTTP }
}

\newglossaryentry{privenetwerk (VPN)}{
name={privé netwerk},
description={een op zichzelf staand netwerk dat alleen gebruikt wordt door een bepaalde instelling, dus volledig gescheiden van het publiekelijk toegankelijk internet}
}

\newglossaryentry{Proces}{
name={Proces},
description={Een applicatie die draait op een host }
}

\newglossaryentry{pulsecodemodulatie (PCM)}{
name={pulsecodemodulatie (PCM)},
description={Door de samplesnelheid en quantiseringswaarden te verhogen => betere benadering van het analoge signaal.}
}

%Q

\newglossaryentry{quantisering }{
name={quantisering },
description={Alle samples worden afgerond naar het dichtstbijzijnde getal in een reeks }
}


%R

\newglossaryentry{Rarest-first}{
name={Rarest-first},
description={Het principe dat gebruikt wordt wanneer een peer een chunk moet gaan halen bij andere peers}
}

\newglossaryentry{Registrar}{
name={Registrar},
description={commerciële entiteit die controleert of een domeinnaam uniek is, de domeinnaam in de DNS plaatst en daar geld voor vraagt.}
}

\newglossaryentry{Round Trip Time (RTT)}{
name={Round Trip Time (RTT)},
description={De tijd die verstrijkt tussen het moment dat een client een verzoek doet om het bestand te ontvangen en het moment dat het wordt ontvangen}
}

\newglossaryentry{RTP (Realtime Transport Protocol)}{
name={RTP (Realtime Transport Protocol)},
description={ staat in gedeelte streamen met UDP}
}

\newglossaryentry{RTP-Header}{
name={RTP-Header},
description={Header voor audio-chunks die bestaat uit het type audiocodering, het volgnummer en de tijdstempel. Deze bestaat uit 12 Bytes.}
}

\newglossaryentry{RTP-pakket. }{
name={RTP-pakket},
description={Audiochunk + RTP Header }
}

%S

\newglossaryentry{Secure Sockets Layer (SSL)}{
name={Secure Sockets Layer (SSL)},
description={Uitgebreide TCP die gebruik maakt van cryptografie}
}

\newglossaryentry{server process}{
name={Server process},
description={process that waits to be contacted }
}

\newglossaryentry{Sessiesleutel}{
name={Sessiesleutel},
description={De gedeelde sleutel die wordt gebruikt om de gegevens zelf te encrypteren. }
}

\newglossaryentry{IDs = Intrusion Detection system }{
name={Signature-based IDS = Intrusion Detection system },
description={ }
}

\newglossaryentry{sessioninitprotocol}{
name={SIP = Session Initiation protocol},
description={ }
}

\newglossaryentry{SMTP }{
name={SMTP},
description={pushprotocol voor mails uit te wisselen}
}

\newglossaryentry{socket}{
name={Socket},
description={een proces  verzendt berichten naar en ontvangt berichten van het netwerk via een netwerkinterface}
}

\newglossaryentry{stateless/toestandsloos protocol}{
name={Stateless/toestandsloos protocol},
description={Omdat een HTTP-server geen info over de clients bijhoudt m.b.t. eerder verzonden berichten}
}

\newglossaryentry{Streamen}{
name={Streamen},
description={ Zorgt ervoor dat de gebruiker een stuk van de video inlaadt en vervolgens de volgende stukjes direct erna inlaadt. Dit voorkomt dat de gebruiker het bestand volledig eerst moet inladen vooraleer het te bekijken}
}

%T


\newglossaryentry{Torrent }{
name={Torrent},
description={de verzameling peers die deelnemen aan de distributie van een bepaald bestand }
}

\newglossaryentry{Tracker }{
name={Tracker },
description={Infrastructuurnode van een torrent}
}

\newglossaryentry{transportmode}{
name={Transport mode IPsec},
description={ }
}


\newglossaryentry{transportlayer}{
name={Transport Layer Security (TLS)},
description={Een iets gewijzigde versie van SSL}
}

\newglossaryentry{tunnel}{
name={Tunnel},
description={ }
}



%U

%V

%W

\newglossaryentry{Webcache (proxyserver)}{
name={Webcache (proxyserver) },
description={Een netwerkentiteit die HTTP-verzoeken afhandelt namens de oorspronkelijke webserver waar het verzoek oorspronkelijk naartoe is gestuurd. }
}

























  %%%%%%acronyms
 
  % label for reference -> acronym itself -> explanation
  \newacronym{acl}{ACL}{Access Control List}
    %\acrlong{acl} Transport Layer Security 
%Displays the phrase which the acronyms stands for. Put the label of the acronym inside the braces. 

%\acrshort{v} TLS
%Prints the acronym whose label is passed as parameter. For instance, \acrshort{gcd} renders as GCD.

%\acrfull{acl }  transport Layer security (TLS)
%Prints both, the acronym and its definition. In the example the output of \acrfull{lcm} is Least Common Multiple (LCM).
 
  % label for reference -> acronym itself -> explanation
  \newacronym{aes}{AES}{Advanced Encryption Standard}
    %\acrlong{aes} Transport Layer Security 
%Displays the phrase which the acronyms stands for. Put the label of the acronym inside the braces. 

%\acrshort{aes} TLS
%Prints the acronym whose label is passed as parameter. For instance, \acrshort{gcd} renders as GCD.

%\acrfull{aes }  transport Layer security (TLS)
%Prints both, the acronym and its definition. In the example the output of \acrfull{lcm} is Least Common Multiple (LCM).
 
  % label for reference -> acronym itself -> explanation
  \newacronym{bgp}{BGP}{Border Gateway Protocol}
    %\acrlong{bgp} Transport Layer Security 
%Displays the phrase which the acronyms stands for. Put the label of the acronym inside the braces. 

%\acrshort{bgp} TLS
%Prints the acronym whose label is passed as parameter. For instance, \acrshort{gcd} renders as GCD.

%\acrfull{bgp }  transport Layer security (TLS)
%Prints both, the acronym and its definition. In the example the output of \acrfull{lcm} is Least Common Multiple (LCM).
 
  % label for reference -> acronym itself -> explanation
  \newacronym{ca}{CA}{certificatie Authoriteit}
    %\acrlong{ca} Transport Layer Security 
%Displays the phrase which the acronyms stands for. Put the label of the acronym inside the braces. 

%\acrshort{ca} TLS
%Prints the acronym whose label is passed as parameter. For instance, \acrshort{gcd} renders as GCD.

%\acrfull{ca }  transport Layer security (TLS)
%Prints both, the acronym and its definition. In the example the output of \acrfull{lcm} is Least Common Multiple (LCM).
 
  % label for reference -> acronym itself -> explanation
  \newacronym{ca}{TTL}{Time To Live}
    %\acrlong{ca} Transport Layer Security 
%Displays the phrase which the acronyms stands for. Put the label of the acronym inside the braces. 

%\acrshort{ca} TLS
%Prints the acronym whose label is passed as parameter. For instance, \acrshort{gcd} renders as GCD.

%\acrfull{ca }  transport Layer security (TLS)
%Prints both, the acronym and its definition. In the example the output of \acrfull{lcm} is Least Common Multiple (LCM).
 
  % label for reference -> acronym itself -> explanation
  \newacronym{cdn}{CDN}{Content distribution networks}
    %\acrlong{cdn} Transport Layer Security 
%Displays the phrase which the acronyms stands for. Put the label of the acronym inside the braces. 

%\acrshort{cdn} TLS
%Prints the acronym whose label is passed as parameter. For instance, \acrshort{gcd} renders as GCD.

%\acrfull{cdn }  transport Layer security (TLS)
%Prints both, the acronym and its definition. In the example the output of \acrfull{lcm} is Least Common Multiple (LCM).
 
  % label for reference -> acronym itself -> explanation
  \newacronym{cbc}{CBC}{Cipher-block chaining}
    %\acrlong{cbc} Transport Layer Security 
%Displays the phrase which the acronyms stands for. Put the label of the acronym inside the braces. 

%\acrshort{cbc} TLS
%Prints the acronym whose label is passed as parameter. For instance, \acrshort{gcd} renders as GCD.

%\acrfull{cbc }  transport Layer security (TLS)
%Prints both, the acronym and its definition. In the example the output of \acrfull{lcm} is Least Common Multiple (LCM).
 
  % label for reference -> acronym itself -> explanation
  \newacronym{dos}{DoS}{denial of service}
    %\acrlong{dos} Transport Layer Security 
%Displays the phrase which the acronyms stands for. Put the label of the acronym inside the braces. 

%\acrshort{dos} TLS
%Prints the acronym whose label is passed as parameter. For instance, \acrshort{gcd} renders as GCD.

%\acrfull{dos }  transport Layer security (TLS)
%Prints both, the acronym and its definition. In the example the output of \acrfull{lcm} is Least Common Multiple (LCM).
 
  % label for reference -> acronym itself -> explanation
  \newacronym{dns}{DNS}{Domain Name System}
    %\acrlong{dns} Transport Layer Security 
%Displays the phrase which the acronyms stands for. Put the label of the acronym inside the braces. 

%\acrshort{dns} TLS
%Prints the acronym whose label is passed as parameter. For instance, \acrshort{gcd} renders as GCD.

%\acrfull{dns }  transport Layer security (TLS)
%Prints both, the acronym and its definition. In the example the output of \acrfull{lcm} is Least Common Multiple (LCM).
 
  % label for reference -> acronym itself -> explanation
  \newacronym{dash}{DASH}{Dynamic Adaptive Streaming over HTTP}
    %\acrlong{dash} Transport Layer Security 
%Displays the phrase which the acronyms stands for. Put the label of the acronym inside the braces. 

%\acrshort{dash} TLS
%Prints the acronym whose label is passed as parameter. For instance, \acrshort{gcd} renders as GCD.

%\acrfull{dash }  transport Layer security (TLS)
%Prints both, the acronym and its definition. In the example the output of \acrfull{lcm} is Least Common Multiple (LCM).
 
  % label for reference -> acronym itself -> explanation
  \newacronym{fec}{FEC}{Forward Error Correction}
    %\acrlong{fec} Transport Layer Security 
%Displays the phrase which the acronyms stands for. Put the label of the acronym inside the braces. 

%\acrshort{fec} TLS
%Prints the acronym whose label is passed as parameter. For instance, \acrshort{gcd} renders as GCD.

%\acrfull{fec }  transport Layer security (TLS)
%Prints both, the acronym and its definition. In the example the output of \acrfull{lcm} is Least Common Multiple (LCM).
 
  % label for reference -> acronym itself -> explanation
  \newacronym{ids}{IDS}{Intrusion Detection system}
    %\acrlong{ids} Transport Layer Security 
%Displays the phrase which the acronyms stands for. Put the label of the acronym inside the braces. 

%\acrshort{ids} TLS
%Prints the acronym whose label is passed as parameter. For instance, \acrshort{gcd} renders as GCD.

%\acrfull{ids }  transport Layer security (TLS)
%Prints both, the acronym and its definition. In the example the output of \acrfull{lcm} is Least Common Multiple (LCM).
 
  % label for reference -> acronym itself -> explanation
  \newacronym{imap}{IMAP}{nternet Mail Access Protocol}
    %\acrlong{imap} Transport Layer Security 
%Displays the phrase which the acronyms stands for. Put the label of the acronym inside the braces. 

%\acrshort{imap} TLS
%Prints the acronym whose label is passed as parameter. For instance, \acrshort{gcd} renders as GCD.

%\acrfull{imap }  transport Layer security (TLS)
%Prints both, the acronym and its definition. In the example the output of \acrfull{lcm} is Least Common Multiple (LCM).
 
  % label for reference -> acronym itself -> explanation
  \newacronym{iv}{IV}{initialisatievector}
    %\acrlong{iv} Transport Layer Security 
%Displays the phrase which the acronyms stands for. Put the label of the acronym inside the braces. 

%\acrshort{iv} TLS
%Prints the acronym whose label is passed as parameter. For instance, \acrshort{gcd} renders as GCD.

%\acrfull{iv }  transport Layer security (TLS)
%Prints both, the acronym and its definition. In the example the output of \acrfull{lcm} is Least Common Multiple (LCM).
 
  % label for reference -> acronym itself -> explanation
  \newacronym{ike}{IKE}{Internet Key Exchange-protoco}
    %\acrlong{ike} Transport Layer Security 
%Displays the phrase which the acronyms stands for. Put the label of the acronym inside the braces. 

%\acrshort{ike} TLS
%Prints the acronym whose label is passed as parameter. For instance, \acrshort{gcd} renders as GCD.

%\acrfull{ike }  transport Layer security (TLS)
%Prints both, the acronym and its definition. In the example the output of \acrfull{lcm} is Least Common Multiple (LCM).
 
  % label for reference -> acronym itself -> explanation
  \newacronym{mac}{MAC}{Message Authentication Code}
    %\acrlong{mac} Transport Layer Security 
%Displays the phrase which the acronyms stands for. Put the label of the acronym inside the braces. 

%\acrshort{mac} TLS
%Prints the acronym whose label is passed as parameter. For instance, \acrshort{gcd} renders as GCD.

%\acrfull{mac }  transport Layer security (TLS)
%Prints both, the acronym and its definition. In the example the output of \acrfull{lcm} is Least Common Multiple (LCM).
 
  % label for reference -> acronym itself -> explanation
  \newacronym{p2p}{P2P}{peer-to-peer}
    %\acrlong{p2p} Transport Layer Security 
%Displays the phrase which the acronyms stands for. Put the label of the acronym inside the braces. 

%\acrshort{p2p} TLS
%Prints the acronym whose label is passed as parameter. For instance, \acrshort{gcd} renders as GCD.

%\acrfull{p2p }  transport Layer security (TLS)
%Prints both, the acronym and its definition. In the example the output of \acrfull{lcm} is Least Common Multiple (LCM).
 
  % label for reference -> acronym itself -> explanation
  \newacronym{pgp}{PGP}{Pretty Good Privacy}
    %\acrlong{pgp} Transport Layer Security 
%Displays the phrase which the acronyms stands for. Put the label of the acronym inside the braces. 

%\acrshort{pgp} TLS
%Prints the acronym whose label is passed as parameter. For instance, \acrshort{gcd} renders as GCD.

%\acrfull{pgp }  transport Layer security (TLS)
%Prints both, the acronym and its definition. In the example the output of \acrfull{lcm} is Least Common Multiple (LCM).
 
  % label for reference -> acronym itself -> explanation
  \newacronym{pop3}{POP3}{Post Office Protocol}
    %\acrlong{pop3} Transport Layer Security 
%Displays the phrase which the acronyms stands for. Put the label of the acronym inside the braces. 

%\acrshort{pop3} TLS
%Prints the acronym whose label is passed as parameter. For instance, \acrshort{gcd} renders as GCD.

%\acrfull{pop3 }  transport Layer security (TLS)
%Prints both, the acronym and its definition. In the example the output of \acrfull{lcm} is Least Common Multiple (LCM).
 
  % label for reference -> acronym itself -> explanation
  \newacronym{pcm}{PCM}{pulsecodemodulatie}
    %\acrlong{pcm} Transport Layer Security 
%Displays the phrase which the acronyms stands for. Put the label of the acronym inside the braces. 

%\acrshort{pcm} TLS
%Prints the acronym whose label is passed as parameter. For instance, \acrshort{gcd} renders as GCD.

%\acrfull{pcm }  transport Layer security (TLS)
%Prints both, the acronym and its definition. In the example the output of \acrfull{lcm} is Least Common Multiple (LCM).
 
  % label for reference -> acronym itself -> explanation
  \newacronym{vpn}{VPN}{Virtual private network}
    %\acrlong{vpn} Transport Layer Security 
%Displays the phrase which the acronyms stands for. Put the label of the acronym inside the braces. 

%\acrshort{vpn} TLS
%Prints the acronym whose label is passed as parameter. For instance, \acrshort{gcd} renders as GCD.

%\acrfull{vpn }  transport Layer security (TLS)
%Prints both, the acronym and its definition. In the example the output of \acrfull{lcm} is Least Common Multiple (LCM).
 
  % label for reference -> acronym itself -> explanation
  \newacronym{rtt}{RTT}{Round Trip Time}
    %\acrlong{rtt} Transport Layer Security 
%Displays the phrase which the acronyms stands for. Put the label of the acronym inside the braces. 

%\acrshort{rtt} TLS
%Prints the acronym whose label is passed as parameter. For instance, \acrshort{gcd} renders as GCD.

%\acrfull{rtt }  transport Layer security (TLS)
%Prints both, the acronym and its definition. In the example the output of \acrfull{lcm} is Least Common Multiple (LCM).
 
  % label for reference -> acronym itself -> explanation
  \newacronym{ssl}{SSL}{Secure Socket Layer}
   %\acrlong{ssl} Transport Layer Security 
%Displays the phrase which the acronyms stands for. Put the label of the acronym inside the braces. 

%\acrshort{ssl} TLS
%Prints the acronym whose label is passed as parameter. For instance, \acrshort{gcd} renders as GCD.

%\acrfull{ssl }  transport Layer security (TLS)
%Prints both, the acronym and its definition. In the example the output of \acrfull{lcm} is Least Common Multiple (LCM).
  
  % label for reference -> acronym itself -> explanation
  \newacronym{sip}{SIP}{Session Initiation protocol}
  %\acrlong{sip} Transport Layer Security 
%Displays the phrase which the acronyms stands for. Put the label of the acronym inside the braces. 

%\acrshort{sip} TLS
%Prints the acronym whose label is passed as parameter. For instance, \acrshort{gcd} renders as GCD.

%\acrfull{sip }  transport Layer security (TLS)
%Prints both, the acronym and its definition. In the example the output of \acrfull{lcm} is Least Common Multiple (LCM).
  

    
  % label for reference -> acronym itself -> explanation
  \newacronym{smtp}{SMTP}{Simple Mail Transfer protocol}
  %\acrlong{smtp} Transport Layer Security 
%Displays the phrase which the acronyms stands for. Put the label of the acronym inside the braces. 

%\acrshort{smtp} TLS
%Prints the acronym whose label is passed as parameter. For instance, \acrshort{gcd} renders as GCD.

%\acrfull{smtp }  transport Layer security (TLS)
%Prints both, the acronym and its definition. In the example the output of \acrfull{lcm} is Least Common Multiple (LCM).
  

  
  % label for reference -> acronym itself -> explanation
  \newacronym{tls}{TLS}{Transport Layer Security}
  
  %\acrlong{tls} Transport Layer Security 
%Displays the phrase which the acronyms stands for. Put the label of the acronym inside the braces. 

%\acrshort{tls} TLS
%Prints the acronym whose label is passed as parameter. For instance, \acrshort{gcd} renders as GCD.

%\acrfull{tls }  transport Layer security (TLS)
%Prints both, the acronym and its definition. In the example the output of \acrfull{lcm} is Least Common Multiple (LCM).
  
  
  % label for reference -> acronym itself -> explanation
 % \newacronym{}{}{}
  
  % label for reference -> acronym itself -> explanation
  %\newacronym{}{}{}
 

